\chapter{Systemdesign}

%sys-doc, arch, design

\section{Automatic information grabbing}

How could we get semantic information from internet sites?

\subsection{Requirements}

\begin{itemize}
	\item The organisation of the different pages (same provider) may be
	dynamic in some way (some tables may be missing on some pages) so exact
	path specification may be difficult
	\item so when using exact path specification we may need different anchors
	(not only the begin of the document). maybe s.th. like headlines or simple
	"`words"'\\
	e.g.: "`search for 'Gewinn' and then //table[1]/tr[2]/td[2]"'
	\item semantic description may be better like:\\
	"`search for 'EPS' and then take all the columns in that row"'
\end{itemize}

\subsection{What tools already exist for that?}

Keywords: screen scraping, web scraping

\begin{itemize}
	\item Effective Web Data Extraction with Standard XML Technologies\\
	\url{http://www10.org/cdrom/papers/102/index.html}
	\item \url{http://iconico.com/DataExtractor/}
	\item \url{http://www.codeproject.com/internet/webcrawler.asp}
	\item DataFerrett\\
	\url{http://dataferrett.census.gov/TheDataWeb/index.html}
	\item \url{http://www.rubyrailways.com/data-extraction-for-web-20-screen-scraping-in-rubyrails}
	\item \url{http://www.screen-scraper.com/}
	\item \url{http://www.iopus.com/imacros/web-scraping.htm}
	\item \url{http://www.theeasybee.com/}
	\item \url{http://www.perl.com/pub/a/2003/01/22/mechanize.html}
	\item \url{http://simile.mit.edu/wiki/Piggy_Bank}
	\item \url{http://blog.screen-scraper.com/2006/03/21/three-common-methods-for-data-extraction/}
	\item \url{http://www.perl.com/pub/a/2006/06/01/fear-api.html}
	\item \url{http://www.iopus.com/imacros/tutorials/java.htm}
	\item \url{http://www.vogel-nest.de/wiki/Main/WebScraping1}
	\item \url{http://de.wikipedia.org/wiki/Screen_Scraping}
	\item \url{http://www.dalkescientific.com/writings/diary/archive/2005/04/21/screen_scraping.html}
	\item \url{http://www.rexx.com/~dkuhlman/quixote_htmlscraping.html}
	%\item \url{}
\end{itemize}

\subsection{Exact path specification}

\begin{itemize}
	\item not useful to get fundamental data
	\item to much paths to specify
	\item pages for different companies at the same financial site are too different
	(some values are missing, some tables are missing)
	\item may be helpful to get current prices of single stock
\end{itemize}

\subsection{Semantic table description}

\begin{itemize}
	\item get tables from HTML into better parsable format
	\item find usable way to describe the content of the table
	(first row is header, first column is header, row has date format, etc)
\end{itemize}

\section{Growth Rate Calculation}

\begin{itemize}
	\item The 4 growth rates can be weighted. So to get a list with best
	companies on top (like in magic formular) we could weight the 4 growth
	rates and put the ROIC on top.
	\item f�r "`Sticker and MOS Calculator"' auf www.ruleoneinvestor.com siehe Seite 218
	\item f�r "`ROI Calculator"' auf www.ruleoneinvestor.com siehe Seite 220
\end{itemize}

\section{Screen scraping}


Effective Web Data Extraction with Standard XML Technologies
- http://www10.org/cdrom/papers/102/index.html

DataExtractor
- http://iconico.com/DataExtractor/
- http://iconico.com/DataExtractor/help.aspx
- Angabe mehrerer URLs, von denen Daten extrahiert werden sollen
- erlaubt Angabe �bergeordneter Ordner und URLs, Daten werden aus
  untergeordneten Dateien/URLs extrahiert
- Spezifikation, was extrahiert werden soll
  - verschiedene Arten Regeln zu definieren
  - verschiedene vordefinierte Regeln (Email, Telefon-Nummern, URLs)
  - "pattern based rule"
    - verwendet regul�re Ausdr�cke um Daten zu finden und zu extrahieren
  - "text based rule"
    - basiert auf exaktem Text und einfach Wildcards (z.B. f?nd: find, fond)
    - "fuzzy search": Limit der �hnlichkeit, "found" findet "bound" mit Limit = 1
  - "HTML Webpage Script"
    - verwendet JavaScript
    - parst DOM um Daten zu finden
    - zum Extrahieren der Daten gibt es zus�tzliche API
    - der Anwender implementiert letztlich einen DOM Parser, um an die zu extrahierenden
      Daten heranzukommen

Web Data Extraction by Crawling using WINHTTP and Document Object (DOM) Instantiation
- http://www.codeproject.com/internet/webcrawler.asp
- verwendet WINHTTP 5 Bibliothek um HTML Seiten mittels HTTP zu downloaden
- erzeugt einen DOM f�r die HTML Seite
- Daten, die extrahiert werden sollen, k�nnen nicht angegeben werden
- es steht lediglich der DOM zur Verf�gung, ein DOM Parser mit Spezifikation, welche
  Daten extrahiert werden sollen w�re hinzuzuf�gen

Data extraction for Web 2.0: Screen scraping in Ruby/Rails
- http://www.rubyrailways.com/data-extraction-for-web-20-screen-scraping-in-rubyrails
- Artikel
- erkl�rt verschiedene Methoden, um Daten zu spezifizieren und zu extrahieren
- String wrappers
  - HTML Seite
- Tree Wrappers
  - DOM parsen
  - Problem: HTML ist nicht immer valides XML
  - Ruby: HTree (HTML Parser) und REXML (XML Parser)
  - HTML nach XML umwandeln und dann XPath Ausdr�cke verwenden zur Spezifikation der Daten
  - Ruby: Hpricot: XPath direkt auf HTML Dokument (Umwandlung von HTML nach XML passiert intern)
- RubyfulSoup
  - m�chtiges screen-scraping Paket
  - verwendet Ruby-like Syntax, um HTML Tags anzusteuern und ihre Daten zu extrahieren
  - Spezifikation der sematischen Daten also im Ruby Code
  - sozusagen "XPath mit Ruby Syntax"
- WWW::Mechanize
  - Fernsteuerung f�r HTML Formulare
  - Programmatisches Ausf�llen von Forumularen
  - Navigation zur Ergebnisseite
  - Extraktion der Ergebnisse
  - Extraktionsspezifikation nicht so m�chtig, wie bei XPath
- scRUBYt!
  - Ruby Web-Extraktions-Framework basierend on Hpricot und Mechanize
  - Support: Navigation von WWW Seiten, F�llen von Formularen, Extraktion von Daten
  - Ruby Syntax f�r Spezifikation der Daten statt XPath
- WATIR
  - "Web Application Testing in Ruby"
  - nutzt Internet Explorer um WWW Seiten zu steuern (Navigation, Formulare ausf�llen)
  - extrahiert Daten


Building a Web Service to Provide Real-Time Stock Quotes
- http://www.dotnetjunkies.com/Article/A3E8CA89-9AA1-4C3B-BCC6-901C92A523E9.dcik
- Beschreibt einen Web-Service, der Real-Time stock quotes liefert
- Daten von Yahoo! Finance
- C-Sharp Beispiel, wie Daten von einer WWW Seite gelesen werden k�nnen (als HTML)
- verwendet hand-geschriebenen Parser um Daten zu extrahieren, Spezifikation der
  semantischen Daten also im Code


- http://www.screen-scraper.com/
- http://www.iopus.com/imacros/web-scraping.htm
- http://www.theeasybee.com/
- http://www.perl.com/pub/a/2003/01/22/mechanize.html
- \url{http://simile.mit.edu/wiki/Piggy_Bank}
- http://blog.screen-scraper.com/2006/03/21/three-common-methods-for-data-extraction/
- http://www.perl.com/pub/a/2006/06/01/fear-api.html
- http://www.iopus.com/imacros/tutorials/java.htm
- http://www.vogel-nest.de/wiki/Main/WebScraping1
- \url{http://de.wikipedia.org/wiki/Screen_Scraping}
- \url{http://www.dalkescientific.com/writings/diary/archive/2005/04/21/screen_scraping.html}
- \url{http://www.rexx.com/~dkuhlman/quixote_htmlscraping.html}

- avoid being blocked when screen scraping
  - http://www.ddj.com/184405712
  
- setting the proxy on webbrowser
  - http://ryanfarley.com/blog/archive/2004/12/23/1330.aspx


C-Sharp und Tray-Icons
- hilfreich f�r ticker app
- sollte klein (RAM) sein und nur notifikationen zeigen
- http://www.mycsharp.de/wbb2/thread.php?threadid=24261
- http://www.developer.com/net/csharp/article.php/3336751
- http://www.codeproject.com/csharp/trayiconmenu01.asp


=================================================================================
 C O D E
=================================================================================

-------------------------------------------------------------------------------------------------------

http://www.dotnetjunkies.com/Article/A3E8CA89-9AA1-4C3B-BCC6-901C92A523E9.dcik

public static string GetPageContent(string url) // http://finance.yahoo.com/q/ecn?s=MSFT
{
    WebRequest wreq;
    WebResponse wres;
    StreamReader sr;
    String content;

    wreq = HttpWebRequest.Create(url);
    wres = wreq.GetResponse();
    sr = new StreamReader(wres.GetResponseStream());
    content = sr.ReadToEnd();
    sr.Close();

    return content;
}

private string ParsePage(string page)
{
    Int32 i;

    i = page.IndexOf("Last Trade:");
    page = page.Substring(i);

    i = page.IndexOf("<b>");
    page = page.Substring(i);

    i = page.IndexOf("</b>");
    page = page.Substring(0,i);

    page = Regex.Replace(page, "<b>","");
    return page;
}

-------------------------------------------------------------------------------------------------------

\section{GuideLines}

\subsection{ErrorHandling}
- all .net event handlers must have a try cache
  - or can we register OnUnhandledException?
  - to make exceptions visible and log it

- service layer throws exception
- BCs throw exceptions
- BFs catch all exception, log it, propagate it to UI
- UI catches all internal exceptions, log it, show to user
  - shows error messages from BFs to user

- MainUI and MainBE need to impl a communication channel esp. for 
  errors and warnings

\subsubsection{Input/parameter validation}

- UI validates obvious errornous inputs
  e.g.: required fields are not filled, number required
  but input is no valid number
  
- BF does no validation at all
  - except runtime stuff, e.g.: BC not found, communication channel closed
  
- BC validates semantic of inputs
  - e.g. stockcatalog does not exist and should not be created
  

\subsection{Others}


sn -k <xyz.snk>





